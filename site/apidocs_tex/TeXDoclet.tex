\documentclass[11pt,a4paper]{report}
\usepackage{color}
\usepackage{ifthen}
\usepackage{makeidx}
\usepackage{ifpdf}
\usepackage[headings]{fullpage}
\ifpdf \usepackage[pdftex, pdfpagemode={UseOutlines},bookmarks,colorlinks,linkcolor={blue},plainpages=false,pdfpagelabels,citecolor={red},breaklinks=true]{hyperref}
  \usepackage[pdftex]{graphicx}
  \pdfcompresslevel=9
  \DeclareGraphicsRule{*}{mps}{*}{}
\else
  \usepackage[dvips]{graphicx}
\fi

\newcommand{\entityintro}[3]{%
  \hbox to \hsize{%
    \vbox{%
      \hbox to .2in{}%
    }%
    {\bf  #1}%
    \dotfill\pageref{#2}%
  }
  \makebox[\hsize]{%
    \parbox{.4in}{}%
    \parbox[l]{5in}{%
      \vspace{1mm}%
      #3%
      \vspace{1mm}%
    }%
  }%
}
\newcommand{\refdefined}[1]{
\expandafter\ifx\csname r@#1\endcsname\relax
\relax\else
{$($in \ref{#1}, page \pageref{#1}$)$}\fi}
\date{\today}
\title{TeXDoclet Java Documentation\bigskip\\ \Large Created with Javadoc TeXDoclet Doclet}
\author{Greg Wonderly nd S{"o}ren Caspersen nd Stefan Marx}
\chardef\textbackslash=`\\
\makeindex
\begin{document}
\maketitle
\sloppy
\addtocontents{toc}{\protect\markboth{Contents}{Contents}}
\tableofcontents
\chapter*{Class Hierarchy}{
\thispagestyle{empty}
\markboth{Class Hierarchy}{Class Hierarchy}
\addcontentsline{toc}{chapter}{Class Hierarchy}
\section*{Classes}
{\raggedright
\hspace{0.0cm} $\bullet$ java.lang.Object {\tiny \refdefined{java.lang.Object}} \\
\hspace{1.0cm} $\bullet$ com.sun.javadoc.Doclet {\tiny \refdefined{com.sun.javadoc.Doclet}} \\
\hspace{2.0cm} $\bullet$ org.stfm.texdoclet.TeXDoclet {\tiny \refdefined{org.stfm.texdoclet.TeXDoclet}} \\
\hspace{1.0cm} $\bullet$ javax.swing.text.html.HTMLEditorKit.ParserCallback {\tiny \refdefined{javax.swing.text.html.HTMLEditorKit.ParserCallback}} \\
\hspace{2.0cm} $\bullet$ org.stfm.texdoclet.HTMLtoLaTeXBackEnd {\tiny \refdefined{org.stfm.texdoclet.HTMLtoLaTeXBackEnd}} \\
\hspace{1.0cm} $\bullet$ org.stfm.texdoclet.ClassHierachy {\tiny \refdefined{org.stfm.texdoclet.ClassHierachy}} \\
\hspace{1.0cm} $\bullet$ org.stfm.texdoclet.HelpOutput {\tiny \refdefined{org.stfm.texdoclet.HelpOutput}} \\
\hspace{1.0cm} $\bullet$ org.stfm.texdoclet.InterfaceHierachy {\tiny \refdefined{org.stfm.texdoclet.InterfaceHierachy}} \\
\hspace{1.0cm} $\bullet$ org.stfm.texdoclet.Package {\tiny \refdefined{org.stfm.texdoclet.Package}} \\
\hspace{1.0cm} $\bullet$ org.stfm.texdoclet.TableInfo {\tiny \refdefined{org.stfm.texdoclet.TableInfo}} \\
\hspace{1.0cm} $\bullet$ org.stfm.texdoclet.TestFilter {\tiny \refdefined{org.stfm.texdoclet.TestFilter}} \\
}
\section*{Interfaces}
\hspace{0.0cm} $\bullet$ org.stfm.texdoclet.ClassFilter {\tiny \refdefined{org.stfm.texdoclet.ClassFilter}} \\
}
\chapter{Package org.stfm.texdoclet}{
\label{org.stfm.texdoclet}\hypertarget{org.stfm.texdoclet}{}
\hskip -.05in
\hbox to \hsize{\textit{ Package Contents\hfil Page}}
\vskip .13in
\hbox{{\bf  Interfaces}}
\entityintro{ClassFilter}{org.stfm.texdoclet.ClassFilter}{This interface can be implemented and a class name provided to the Doclet to filter which classes are and are not included in the output document.}
\vskip .13in
\hbox{{\bf  Classes}}
\entityintro{ClassHierachy}{org.stfm.texdoclet.ClassHierachy}{Manages and prints a class hierarchy.}
\entityintro{HelpOutput}{org.stfm.texdoclet.HelpOutput}{}
\entityintro{HTMLtoLaTeXBackEnd}{org.stfm.texdoclet.HTMLtoLaTeXBackEnd}{

This class implements a \texttt{\small ParserCallback} that translates HTML to the corresponding \LaTeX{}.}
\entityintro{InterfaceHierachy}{org.stfm.texdoclet.InterfaceHierachy}{Manages and prints a interface hierarchy.}
\entityintro{Package}{org.stfm.texdoclet.Package}{This class is used to manage the contents of a Java package.}
\entityintro{TableInfo}{org.stfm.texdoclet.TableInfo}{This class provides support for converting HTML tables into \LaTeX{} tables.}
\entityintro{TestFilter}{org.stfm.texdoclet.TestFilter}{This class filters out classes beginning with "Test" when applied to the Doclet.}
\entityintro{TeXDoclet}{org.stfm.texdoclet.TeXDoclet}{This class provides a Java \texttt{\small javadoc} Doclet which generates a \LaTeXe{} document out of the java classes that it is used on.}
\vskip .1in
\vskip .1in
This doclet is based on the doclet originally created by Greg Wonderly of \hyperref{http://www.c2-tech.com}{}{}{C2 technologies Inc.} and its revision by \hyperref{http://www.xosoftware.dk}{}{}{XO Software}. The project of Greg Wonderly is available here : \hyperref{http://java.net/projects/texdoclet}{}{}{http://java.net/projects/texdoclet}.\section{\label{org.stfm.texdoclet.ClassFilter}\index{ClassFilter@\textit{ ClassFilter}}Interface ClassFilter}{
\hypertarget{org.stfm.texdoclet.ClassFilter}{}\vskip .1in 
This interface can be implemented and a class name provided to the Doclet to filter which classes are and are not included in the output document.\vskip .1in 
\subsection{Declaration}{
\small public interface ClassFilter
}
\subsection{All known subinterfaces}{TestFilter\small{\refdefined{org.stfm.texdoclet.TestFilter}}}
\subsection{All classes known to implement interface}{TestFilter\small{\refdefined{org.stfm.texdoclet.TestFilter}}}
\subsection{Method summary}{
\begin{verse}
\hyperlink{org.stfm.texdoclet.ClassFilter.includeClass(com.sun.javadoc.ClassDoc)}{{\bf includeClass(ClassDoc)}} Filters the ClassDoc passed.\\
\end{verse}
}
\subsection{Methods}{
\vskip -2em
\begin{itemize}
\item{ 
\index{includeClass(ClassDoc)}
\hypertarget{org.stfm.texdoclet.ClassFilter.includeClass(com.sun.javadoc.ClassDoc)}{{\bf  includeClass}\\}
\texttt{ boolean\ {\bf  includeClass}(\texttt{com.sun.javadoc.ClassDoc} {\bf  cd})
\label{org.stfm.texdoclet.ClassFilter.includeClass(com.sun.javadoc.ClassDoc)}}%end signature
\begin{itemize}
\item{
{\bf  Description}

Filters the ClassDoc passed. If true is returned, the passed class will be included into the output. If false is returned, this document will not be included.
}
\end{itemize}
}%end item
\end{itemize}
}
}
\section{\label{org.stfm.texdoclet.ClassHierachy}\index{ClassHierachy}Class ClassHierachy}{
\hypertarget{org.stfm.texdoclet.ClassHierachy}{}\vskip .1in 
Manages and prints a class hierarchy. Use \texttt{\small add} to add another class to the hierarchy. Use \texttt{\small printTree} to print the corresponding \LaTeX{}.\vskip .1in 
\subsection{Declaration}{
\small public class ClassHierachy
\\ {\bf  extends} java.lang.Object
\refdefined{java.lang.Object}}
\subsection{Field summary}{
\begin{verse}
\hyperlink{org.stfm.texdoclet.ClassHierachy.root}{{\bf root}} \\
\end{verse}
}
\subsection{Constructor summary}{
\begin{verse}
\hyperlink{org.stfm.texdoclet.ClassHierachy()}{{\bf ClassHierachy()}} Creates new ClassHierachy\\
\end{verse}
}
\subsection{Method summary}{
\begin{verse}
\hyperlink{org.stfm.texdoclet.ClassHierachy.add(com.sun.javadoc.ClassDoc)}{{\bf add(ClassDoc)}} Adds another class to the hierachy\\
\hyperlink{org.stfm.texdoclet.ClassHierachy.printBranch(com.sun.javadoc.RootDoc, java.util.SortedMap, double, double)}{{\bf printBranch(RootDoc, SortedMap, double, double)}} Prints a branch of the tree.\\
\hyperlink{org.stfm.texdoclet.ClassHierachy.printTree(com.sun.javadoc.RootDoc, double)}{{\bf printTree(RootDoc, double)}} Prints the \LaTeX{} corresponding to the tree.\\
\end{verse}
}
\subsection{Fields}{
\begin{itemize}
\item{
\index{root}
\label{org.stfm.texdoclet.ClassHierachy.root}\hypertarget{org.stfm.texdoclet.ClassHierachy.root}{public java.util.SortedMap {\bf  root}}
}
\end{itemize}
}
\subsection{Constructors}{
\vskip -2em
\begin{itemize}
\item{ 
\index{ClassHierachy()}
\hypertarget{org.stfm.texdoclet.ClassHierachy()}{{\bf  ClassHierachy}\\}
\texttt{public\ {\bf  ClassHierachy}()
\label{org.stfm.texdoclet.ClassHierachy()}}%end signature
\begin{itemize}
\item{
{\bf  Description}

Creates new ClassHierachy
}
\end{itemize}
}%end item
\end{itemize}
}
\subsection{Methods}{
\vskip -2em
\begin{itemize}
\item{ 
\index{add(ClassDoc)}
\hypertarget{org.stfm.texdoclet.ClassHierachy.add(com.sun.javadoc.ClassDoc)}{{\bf  add}\\}
\texttt{protected java.util.SortedMap\ {\bf  add}(\texttt{com.sun.javadoc.ClassDoc} {\bf  cls})
\label{org.stfm.texdoclet.ClassHierachy.add(com.sun.javadoc.ClassDoc)}}%end signature
\begin{itemize}
\item{
{\bf  Description}

Adds another class to the hierachy
}
\end{itemize}
}%end item
\item{ 
\index{printBranch(RootDoc, SortedMap, double, double)}
\hypertarget{org.stfm.texdoclet.ClassHierachy.printBranch(com.sun.javadoc.RootDoc, java.util.SortedMap, double, double)}{{\bf  printBranch}\\}
\texttt{protected void\ {\bf  printBranch}(\texttt{com.sun.javadoc.RootDoc} {\bf  rootDoc},
\texttt{java.util.SortedMap} {\bf  map},
\texttt{double} {\bf  indent},
\texttt{double} {\bf  overviewindent})
\label{org.stfm.texdoclet.ClassHierachy.printBranch(com.sun.javadoc.RootDoc, java.util.SortedMap, double, double)}}%end signature
\begin{itemize}
\item{
{\bf  Description}

Prints a branch of the tree. The branch is printed using \texttt{\small TeXDoclet.os}.
}
\end{itemize}
}%end item
\item{ 
\index{printTree(RootDoc, double)}
\hypertarget{org.stfm.texdoclet.ClassHierachy.printTree(com.sun.javadoc.RootDoc, double)}{{\bf  printTree}\\}
\texttt{public void\ {\bf  printTree}(\texttt{com.sun.javadoc.RootDoc} {\bf  rootDoc},
\texttt{double} {\bf  overviewindent})
\label{org.stfm.texdoclet.ClassHierachy.printTree(com.sun.javadoc.RootDoc, double)}}%end signature
\begin{itemize}
\item{
{\bf  Description}

Prints the \LaTeX{} corresponding to the tree. The tree is printed using \texttt{\small TeXDoclet.os}.
}
\end{itemize}
}%end item
\end{itemize}
}
}
\section{\label{org.stfm.texdoclet.HelpOutput}\index{HelpOutput}Class HelpOutput}{
\hypertarget{org.stfm.texdoclet.HelpOutput}{}\vskip .1in 
\subsection{Declaration}{
\small public class HelpOutput
\\ {\bf  extends} java.lang.Object
\refdefined{java.lang.Object}}
\subsection{Constructor summary}{
\begin{verse}
\hyperlink{org.stfm.texdoclet.HelpOutput()}{{\bf HelpOutput()}} \\
\end{verse}
}
\subsection{Method summary}{
\begin{verse}
\hyperlink{org.stfm.texdoclet.HelpOutput.printHelp()}{{\bf printHelp()}} \\
\end{verse}
}
\subsection{Constructors}{
\vskip -2em
\begin{itemize}
\item{ 
\index{HelpOutput()}
\hypertarget{org.stfm.texdoclet.HelpOutput()}{{\bf  HelpOutput}\\}
\texttt{public\ {\bf  HelpOutput}()
\label{org.stfm.texdoclet.HelpOutput()}}%end signature
}%end item
\end{itemize}
}
\subsection{Methods}{
\vskip -2em
\begin{itemize}
\item{ 
\index{printHelp()}
\hypertarget{org.stfm.texdoclet.HelpOutput.printHelp()}{{\bf  printHelp}\\}
\texttt{protected static void\ {\bf  printHelp}()
\label{org.stfm.texdoclet.HelpOutput.printHelp()}}%end signature
}%end item
\end{itemize}
}
}
\section{\label{org.stfm.texdoclet.HTMLtoLaTeXBackEnd}\index{HTMLtoLaTeXBackEnd}Class HTMLtoLaTeXBackEnd}{
\hypertarget{org.stfm.texdoclet.HTMLtoLaTeXBackEnd}{}\vskip .1in 


This class implements a \texttt{\small ParserCallback} that translates HTML to the corresponding \LaTeX{}. Not all tags a processed but the most common are.

HTML links to files located in the doc-files directory ({appendix\_a.html\refdefined{appendix1}}, {appendix\_b.txt\refdefined{appendix2}}) are transformed to references to the appendix, whereby the referenced files itself are included in the appendix.\vskip .1in 
\subsection{See also}{}

  \begin{list}{-- }{\setlength{\itemsep}{0cm}\setlength{\parsep}{0cm}}
\item{ \texttt{\hyperlink{javax.swing.text.html.parser.ParserDelegator}{javax.swing.text.html.parser.ParserDelegator}} {\small 
\refdefined{javax.swing.text.html.parser.ParserDelegator}}%end
} 
  \end{list}
\subsection{Declaration}{
\small public class HTMLtoLaTeXBackEnd
\\ {\bf  extends} javax.swing.text.html.HTMLEditorKit.ParserCallback
\refdefined{javax.swing.text.html.HTMLEditorKit.ParserCallback}}
\subsection{Constructor summary}{
\begin{verse}
\hyperlink{org.stfm.texdoclet.HTMLtoLaTeXBackEnd(java.lang.StringBuffer)}{{\bf HTMLtoLaTeXBackEnd(StringBuffer)}} Constructs a new instance.\\
\end{verse}
}
\subsection{Method summary}{
\begin{verse}
\hyperlink{org.stfm.texdoclet.HTMLtoLaTeXBackEnd.fixText(java.lang.String)}{{\bf fixText(String)}} Converts a HTML string into \LaTeX{} using an instance of \texttt{\small HTMLtoLaTeXBackEnd}.\\
\hyperlink{org.stfm.texdoclet.HTMLtoLaTeXBackEnd.handleEndTag(javax.swing.text.html.HTML.Tag, int)}{{\bf handleEndTag(HTML.Tag, int)}} This method handles HTML tags that mark an ending (e.g.\\
\hyperlink{org.stfm.texdoclet.HTMLtoLaTeXBackEnd.handleSimpleTag(javax.swing.text.html.HTML.Tag, javax.swing.text.MutableAttributeSet, int)}{{\bf handleSimpleTag(HTML.Tag, MutableAttributeSet, int)}} This method handles simple HTML tags (e.g.\\
\hyperlink{org.stfm.texdoclet.HTMLtoLaTeXBackEnd.handleStartTag(javax.swing.text.html.HTML.Tag, javax.swing.text.MutableAttributeSet, int)}{{\bf handleStartTag(HTML.Tag, MutableAttributeSet, int)}} This method handles HTML tags that mark a beginning (e.g.\\
\hyperlink{org.stfm.texdoclet.HTMLtoLaTeXBackEnd.handleText(char[], int)}{{\bf handleText(char\lbrack \rbrack , int)}} This method handles all other text.\\
\end{verse}
}
\subsection{Constructors}{
\vskip -2em
\begin{itemize}
\item{ 
\index{HTMLtoLaTeXBackEnd(StringBuffer)}
\hypertarget{org.stfm.texdoclet.HTMLtoLaTeXBackEnd(java.lang.StringBuffer)}{{\bf  HTMLtoLaTeXBackEnd}\\}
\texttt{public\ {\bf  HTMLtoLaTeXBackEnd}(\texttt{java.lang.StringBuffer} {\bf  stringBuffer})
\label{org.stfm.texdoclet.HTMLtoLaTeXBackEnd(java.lang.StringBuffer)}}%end signature
\begin{itemize}
\item{
{\bf  Description}

Constructs a new instance.
}
\item{
{\bf  Parameters}
  \begin{itemize}
   \item{
\texttt{stringBuffer} -- The \texttt{\small StringBuffer} where the translated HTML is appended.}
  \end{itemize}
}%end item
\end{itemize}
}%end item
\end{itemize}
}
\subsection{Methods}{
\vskip -2em
\begin{itemize}
\item{ 
\index{fixText(String)}
\hypertarget{org.stfm.texdoclet.HTMLtoLaTeXBackEnd.fixText(java.lang.String)}{{\bf  fixText}\\}
\texttt{public static java.lang.String\ {\bf  fixText}(\texttt{java.lang.String} {\bf  str})
\label{org.stfm.texdoclet.HTMLtoLaTeXBackEnd.fixText(java.lang.String)}}%end signature
\begin{itemize}
\item{
{\bf  Description}

Converts a HTML string into \LaTeX{} using an instance of \texttt{\small HTMLtoLaTeXBackEnd}.
}
\item{{\bf  Returns} -- 
The converted string. 
}%end item
\end{itemize}
}%end item
\item{ 
\index{handleEndTag(HTML.Tag, int)}
\hypertarget{org.stfm.texdoclet.HTMLtoLaTeXBackEnd.handleEndTag(javax.swing.text.html.HTML.Tag, int)}{{\bf  handleEndTag}\\}
\texttt{public void\ {\bf  handleEndTag}(\texttt{javax.swing.text.html.HTML.Tag} {\bf  tag},
\texttt{int} {\bf  pos})
\label{org.stfm.texdoclet.HTMLtoLaTeXBackEnd.handleEndTag(javax.swing.text.html.HTML.Tag, int)}}%end signature
\begin{itemize}
\item{
{\bf  Description}

This method handles HTML tags that mark an ending (e.g. \texttt{\small \textless /P\textgreater }-tags). It is called by the parser whenever such a tag is encountered.
}
\end{itemize}
}%end item
\item{ 
\index{handleSimpleTag(HTML.Tag, MutableAttributeSet, int)}
\hypertarget{org.stfm.texdoclet.HTMLtoLaTeXBackEnd.handleSimpleTag(javax.swing.text.html.HTML.Tag, javax.swing.text.MutableAttributeSet, int)}{{\bf  handleSimpleTag}\\}
\texttt{public void\ {\bf  handleSimpleTag}(\texttt{javax.swing.text.html.HTML.Tag} {\bf  tag},
\texttt{javax.swing.text.MutableAttributeSet} {\bf  attrSet},
\texttt{int} {\bf  pos})
\label{org.stfm.texdoclet.HTMLtoLaTeXBackEnd.handleSimpleTag(javax.swing.text.html.HTML.Tag, javax.swing.text.MutableAttributeSet, int)}}%end signature
\begin{itemize}
\item{
{\bf  Description}

This method handles simple HTML tags (e.g. \texttt{\small \textless HR\textgreater }-tags). It is called by the parser whenever such a tag is encountered.
}
\end{itemize}
}%end item
\item{ 
\index{handleStartTag(HTML.Tag, MutableAttributeSet, int)}
\hypertarget{org.stfm.texdoclet.HTMLtoLaTeXBackEnd.handleStartTag(javax.swing.text.html.HTML.Tag, javax.swing.text.MutableAttributeSet, int)}{{\bf  handleStartTag}\\}
\texttt{public void\ {\bf  handleStartTag}(\texttt{javax.swing.text.html.HTML.Tag} {\bf  tag},
\texttt{javax.swing.text.MutableAttributeSet} {\bf  attrSet},
\texttt{int} {\bf  pos})
\label{org.stfm.texdoclet.HTMLtoLaTeXBackEnd.handleStartTag(javax.swing.text.html.HTML.Tag, javax.swing.text.MutableAttributeSet, int)}}%end signature
\begin{itemize}
\item{
{\bf  Description}

This method handles HTML tags that mark a beginning (e.g. \texttt{\small \textless P\textgreater }-tags). It is called by the parser whenever such a tag is encountered.
}
\end{itemize}
}%end item
\item{ 
\index{handleText(char\lbrack \rbrack , int)}
\hypertarget{org.stfm.texdoclet.HTMLtoLaTeXBackEnd.handleText(char[], int)}{{\bf  handleText}\\}
\texttt{public void\ {\bf  handleText}(\texttt{char\lbrack \rbrack } {\bf  data},
\texttt{int} {\bf  pos})
\label{org.stfm.texdoclet.HTMLtoLaTeXBackEnd.handleText(char[], int)}}%end signature
\begin{itemize}
\item{
{\bf  Description}

This method handles all other text.
}
\end{itemize}
}%end item
\end{itemize}
}
\subsection{Members inherited from class HTMLEditorKit.ParserCallback }{
\texttt{javax.swing.text.html.HTMLEditorKit.ParserCallback} {\small 
\refdefined{javax.swing.text.html.HTMLEditorKit.ParserCallback}}
{\small 

flush, handleComment, handleEndOfLineString, handleEndTag, handleError, handleSimpleTag, handleStartTag, handleText, IMPLIED}
}
\section{\label{org.stfm.texdoclet.InterfaceHierachy}\index{InterfaceHierachy}Class InterfaceHierachy}{
\hypertarget{org.stfm.texdoclet.InterfaceHierachy}{}\vskip .1in 
Manages and prints a interface hierarchy. Use \texttt{\small add} to add another interface to the hierarchy. Use \texttt{\small printTree} to print the corresponding \LaTeX{}.\vskip .1in 
\subsection{Declaration}{
\small public class InterfaceHierachy
\\ {\bf  extends} java.lang.Object
\refdefined{java.lang.Object}}
\subsection{Field summary}{
\begin{verse}
\hyperlink{org.stfm.texdoclet.InterfaceHierachy.root}{{\bf root}} \\
\end{verse}
}
\subsection{Constructor summary}{
\begin{verse}
\hyperlink{org.stfm.texdoclet.InterfaceHierachy()}{{\bf InterfaceHierachy()}} Creates new InterfaceHierachy\\
\end{verse}
}
\subsection{Method summary}{
\begin{verse}
\hyperlink{org.stfm.texdoclet.InterfaceHierachy.add(com.sun.javadoc.ClassDoc)}{{\bf add(ClassDoc)}} Adds another interface to the hierachy\\
\hyperlink{org.stfm.texdoclet.InterfaceHierachy.printBranch(com.sun.javadoc.RootDoc, java.util.SortedMap, double, double)}{{\bf printBranch(RootDoc, SortedMap, double, double)}} Prints a branch of the tree.\\
\hyperlink{org.stfm.texdoclet.InterfaceHierachy.printTree(com.sun.javadoc.RootDoc, double)}{{\bf printTree(RootDoc, double)}} Prints the \LaTeX{} corresponding to the tree.\\
\end{verse}
}
\subsection{Fields}{
\begin{itemize}
\item{
\index{root}
\label{org.stfm.texdoclet.InterfaceHierachy.root}\hypertarget{org.stfm.texdoclet.InterfaceHierachy.root}{public java.util.SortedMap {\bf  root}}
}
\end{itemize}
}
\subsection{Constructors}{
\vskip -2em
\begin{itemize}
\item{ 
\index{InterfaceHierachy()}
\hypertarget{org.stfm.texdoclet.InterfaceHierachy()}{{\bf  InterfaceHierachy}\\}
\texttt{public\ {\bf  InterfaceHierachy}()
\label{org.stfm.texdoclet.InterfaceHierachy()}}%end signature
\begin{itemize}
\item{
{\bf  Description}

Creates new InterfaceHierachy
}
\end{itemize}
}%end item
\end{itemize}
}
\subsection{Methods}{
\vskip -2em
\begin{itemize}
\item{ 
\index{add(ClassDoc)}
\hypertarget{org.stfm.texdoclet.InterfaceHierachy.add(com.sun.javadoc.ClassDoc)}{{\bf  add}\\}
\texttt{protected java.util.SortedMap\ {\bf  add}(\texttt{com.sun.javadoc.ClassDoc} {\bf  cls})
\label{org.stfm.texdoclet.InterfaceHierachy.add(com.sun.javadoc.ClassDoc)}}%end signature
\begin{itemize}
\item{
{\bf  Description}

Adds another interface to the hierachy
}
\end{itemize}
}%end item
\item{ 
\index{printBranch(RootDoc, SortedMap, double, double)}
\hypertarget{org.stfm.texdoclet.InterfaceHierachy.printBranch(com.sun.javadoc.RootDoc, java.util.SortedMap, double, double)}{{\bf  printBranch}\\}
\texttt{protected void\ {\bf  printBranch}(\texttt{com.sun.javadoc.RootDoc} {\bf  rootDoc},
\texttt{java.util.SortedMap} {\bf  map},
\texttt{double} {\bf  indent},
\texttt{double} {\bf  overviewindent})
\label{org.stfm.texdoclet.InterfaceHierachy.printBranch(com.sun.javadoc.RootDoc, java.util.SortedMap, double, double)}}%end signature
\begin{itemize}
\item{
{\bf  Description}

Prints a branch of the tree. The branch is printed using \texttt{\small TeXDoclet.os}.
}
\end{itemize}
}%end item
\item{ 
\index{printTree(RootDoc, double)}
\hypertarget{org.stfm.texdoclet.InterfaceHierachy.printTree(com.sun.javadoc.RootDoc, double)}{{\bf  printTree}\\}
\texttt{public void\ {\bf  printTree}(\texttt{com.sun.javadoc.RootDoc} {\bf  rootDoc},
\texttt{double} {\bf  overviewindent})
\label{org.stfm.texdoclet.InterfaceHierachy.printTree(com.sun.javadoc.RootDoc, double)}}%end signature
\begin{itemize}
\item{
{\bf  Description}

Prints the \LaTeX{} corresponding to the tree. The tree is printed using \texttt{\small TeXDoclet.os}.
}
\end{itemize}
}%end item
\end{itemize}
}
}
\section{\label{org.stfm.texdoclet.Package}\index{Package}Class Package}{
\hypertarget{org.stfm.texdoclet.Package}{}\vskip .1in 
This class is used to manage the contents of a Java package. It accepts ClassDoc objects and examines them and groups them according to whether they are classes, interfaces, exceptions or errors. The accumulated Vectors can then be processed to get to all of the elements of the package that fall into each category. If needed the classes, interfaces, exceptions and errors can be sorted using the \texttt{\small sort} method.\vskip .1in 
\subsection{See also}{}

  \begin{list}{-- }{\setlength{\itemsep}{0cm}\setlength{\parsep}{0cm}}
\item{ \texttt{\hyperlink{org.stfm.texdoclet.Package.sort()}{Package.sort()}} {\small 
\refdefined{org.stfm.texdoclet.Package.sort()}}%end
} 
  \end{list}
\subsection{Declaration}{
\small public class Package
\\ {\bf  extends} java.lang.Object
\refdefined{java.lang.Object}}
\subsection{Field summary}{
\begin{verse}
\hyperlink{org.stfm.texdoclet.Package.classes}{{\bf classes}} The classes this package has in it\\
\hyperlink{org.stfm.texdoclet.Package.errors}{{\bf errors}} The errors this package has in it\\
\hyperlink{org.stfm.texdoclet.Package.exceptions}{{\bf exceptions}} The exceptions this package has in it\\
\hyperlink{org.stfm.texdoclet.Package.interfaces}{{\bf interfaces}} The interfaces this package has in it\\
\hyperlink{org.stfm.texdoclet.Package.pkg}{{\bf pkg}} The name of the package this object is for\\
\hyperlink{org.stfm.texdoclet.Package.pkgDoc}{{\bf pkgDoc}} \\
\end{verse}
}
\subsection{Constructor summary}{
\begin{verse}
\hyperlink{org.stfm.texdoclet.Package(java.lang.String, com.sun.javadoc.PackageDoc)}{{\bf Package(String, PackageDoc)}} Construct a new object corresponding to the passed package name.\\
\end{verse}
}
\subsection{Method summary}{
\begin{verse}
\hyperlink{org.stfm.texdoclet.Package.addElement(com.sun.javadoc.ClassDoc)}{{\bf addElement(ClassDoc)}} Adds a ClassDoc element to this package.\\
\hyperlink{org.stfm.texdoclet.Package.sort()}{{\bf sort()}} Sorts the vectors of classes, interfaces exceptions and errors.\\
\end{verse}
}
\subsection{Fields}{
\begin{itemize}
\item{
\index{pkgDoc}
\label{org.stfm.texdoclet.Package.pkgDoc}\hypertarget{org.stfm.texdoclet.Package.pkgDoc}{protected com.sun.javadoc.PackageDoc {\bf  pkgDoc}}
}
\item{
\index{pkg}
\label{org.stfm.texdoclet.Package.pkg}\hypertarget{org.stfm.texdoclet.Package.pkg}{protected java.lang.String {\bf  pkg}}
\begin{itemize}
\item{\vskip -.9ex 
The name of the package this object is for}
\end{itemize}
}
\item{
\index{classes}
\label{org.stfm.texdoclet.Package.classes}\hypertarget{org.stfm.texdoclet.Package.classes}{protected java.util.Vector {\bf  classes}}
\begin{itemize}
\item{\vskip -.9ex 
The classes this package has in it}
\end{itemize}
}
\item{
\index{interfaces}
\label{org.stfm.texdoclet.Package.interfaces}\hypertarget{org.stfm.texdoclet.Package.interfaces}{protected java.util.Vector {\bf  interfaces}}
\begin{itemize}
\item{\vskip -.9ex 
The interfaces this package has in it}
\end{itemize}
}
\item{
\index{exceptions}
\label{org.stfm.texdoclet.Package.exceptions}\hypertarget{org.stfm.texdoclet.Package.exceptions}{protected java.util.Vector {\bf  exceptions}}
\begin{itemize}
\item{\vskip -.9ex 
The exceptions this package has in it}
\end{itemize}
}
\item{
\index{errors}
\label{org.stfm.texdoclet.Package.errors}\hypertarget{org.stfm.texdoclet.Package.errors}{protected java.util.Vector {\bf  errors}}
\begin{itemize}
\item{\vskip -.9ex 
The errors this package has in it}
\end{itemize}
}
\end{itemize}
}
\subsection{Constructors}{
\vskip -2em
\begin{itemize}
\item{ 
\index{Package(String, PackageDoc)}
\hypertarget{org.stfm.texdoclet.Package(java.lang.String, com.sun.javadoc.PackageDoc)}{{\bf  Package}\\}
\texttt{public\ {\bf  Package}(\texttt{java.lang.String} {\bf  pkg},
\texttt{com.sun.javadoc.PackageDoc} {\bf  doc})
\label{org.stfm.texdoclet.Package(java.lang.String, com.sun.javadoc.PackageDoc)}}%end signature
\begin{itemize}
\item{
{\bf  Description}

Construct a new object corresponding to the passed package name.
}
\item{
{\bf  Parameters}
  \begin{itemize}
   \item{
\texttt{pkg} -- the package name to use}
  \end{itemize}
}%end item
\end{itemize}
}%end item
\end{itemize}
}
\subsection{Methods}{
\vskip -2em
\begin{itemize}
\item{ 
\index{addElement(ClassDoc)}
\hypertarget{org.stfm.texdoclet.Package.addElement(com.sun.javadoc.ClassDoc)}{{\bf  addElement}\\}
\texttt{public void\ {\bf  addElement}(\texttt{com.sun.javadoc.ClassDoc} {\bf  cd})
\label{org.stfm.texdoclet.Package.addElement(com.sun.javadoc.ClassDoc)}}%end signature
\begin{itemize}
\item{
{\bf  Description}

Adds a ClassDoc element to this package.
}
\item{
{\bf  Parameters}
  \begin{itemize}
   \item{
\texttt{cd} -- the object to add to this package}
  \end{itemize}
}%end item
\end{itemize}
}%end item
\item{ 
\index{sort()}
\hypertarget{org.stfm.texdoclet.Package.sort()}{{\bf  sort}\\}
\texttt{public void\ {\bf  sort}()
\label{org.stfm.texdoclet.Package.sort()}}%end signature
\begin{itemize}
\item{
{\bf  Description}

Sorts the vectors of classes, interfaces exceptions and errors.
}
\end{itemize}
}%end item
\end{itemize}
}
}
\section{\label{org.stfm.texdoclet.TableInfo}\index{TableInfo}Class TableInfo}{
\hypertarget{org.stfm.texdoclet.TableInfo}{}\vskip .1in 
This class provides support for converting HTML tables into \LaTeX{} tables. Some of the things {\bf NOT} implemented include the following:\begin{itemize}
\item{\vskip -.8ex valign attributes are not processed, but align= is.}
\item{\vskip -.8ex rowspan attributes are not processed, but colspan= is.}
\item{\vskip -.8ex the argument to border= in the table tag is not used to control line size}
\end{itemize}
\mbox{}\newline Here is an example table.


% Table #0
\begin{center}
\newlength{\tblaaacaaaw}
\setlength{\tblaaacaaaw}{0.3\linewidth}
\newlength{\tblaaacaabw}
\setlength{\tblaaacaabw}{0.3\linewidth}
\newlength{\tblaaacaacw}
\setlength{\tblaaacaacw}{0.3\linewidth}
\colorbox[rgb]{0.8666666666666667,0.8666666666666667,0.8666666666666667}{\begin{tabular}{|p{\tblaaacaaaw}|p{\tblaaacaabw}|p{\tblaaacaacw}|}
 \hline {\bf Column 1 Heading} & {\bf Column two heading} & {\bf Column three heading} \\ \hline
{data} & \multicolumn{2}{p{\tblaaacaabw}|}{Span two columns} \\ \hline
{\textit{ more data}} & \multicolumn{1}{r|}{right} & \multicolumn{1}{p{\tblaaacaacw}|}{left} \\ \hline
\multicolumn{3}{|p{\tblaaacaaaw}|}{
% Table #1
\begin{center}
\newlength{\tblaabcaaaw}
\setlength{\tblaabcaaaw}{0.3\linewidth}
\newlength{\tblaabcaabw}
\setlength{\tblaabcaabw}{0.3\linewidth}
\newlength{\tblaabcaacw}
\setlength{\tblaabcaacw}{0.3\linewidth}
\colorbox[rgb]{0.9333333333333333,0.9333333333333333,0.9333333333333333}{\begin{tabular}{|p{\tblaabcaaaw}|p{\tblaabcaabw}|p{\tblaabcaacw}|}
 \hline \multicolumn{3}{|p{\tblaabcaaaw}|}{\bf A nested table example} \\ \hline
{\bf Column one Heading} & {\bf Column two heading} & {\bf Column three heading} \\ \hline
{data} & \multicolumn{2}{p{\tblaabcaabw}|}{Span two columns} \\ \hline
{\textit{ more data}} & \multicolumn{1}{r|}{right} & \multicolumn{1}{p{\tblaabcaacw}|}{left} \\ \hline
{\texttt{\small
\mbox{}\newline \phantom{ }\phantom{ }1}\mbox{}\newline
\texttt{\small \phantom{ }\phantom{ }\phantom{ }2}\mbox{}\newline
\texttt{\small \phantom{ }\phantom{ }3}\mbox{}\newline
\texttt{\small \phantom{ }\phantom{ }\phantom{ }\phantom{ }4}\mbox{}\newline
\texttt{\small \phantom{ }}
} & {\texttt{\small
\mbox{}\newline \phantom{ }\phantom{ }first\phantom{ }line}\mbox{}\newline
\texttt{\small \phantom{ }\phantom{ }second\phantom{ }line}\mbox{}\newline
\texttt{\small \phantom{ }\phantom{ }third\phantom{ }line}\mbox{}\newline
\texttt{\small \phantom{ }\phantom{ }fourth\phantom{ }line}\mbox{}\newline
\texttt{\small \phantom{ }}
} \\ \hline
\end{tabular}
}
\end{center}
} \\ \hline
\end{tabular}
}
\end{center}
\vskip .1in 
\subsection{Declaration}{
\small public class TableInfo
\\ {\bf  extends} java.lang.Object
\refdefined{java.lang.Object}}
\subsection{Constructor summary}{
\begin{verse}
\hyperlink{org.stfm.texdoclet.TableInfo()}{{\bf TableInfo()}} \\
\end{verse}
}
\subsection{Method summary}{
\begin{verse}
\hyperlink{org.stfm.texdoclet.TableInfo.endCol()}{{\bf endCol()}} Ends the current column.\\
\hyperlink{org.stfm.texdoclet.TableInfo.endRow()}{{\bf endRow()}} Ends the current row.\\
\hyperlink{org.stfm.texdoclet.TableInfo.endTable()}{{\bf endTable()}} Ends the table, closing the last row as needed\\
\hyperlink{org.stfm.texdoclet.TableInfo.startCol(javax.swing.text.MutableAttributeSet)}{{\bf startCol(MutableAttributeSet)}} Starts a new column, possibly closing the current column if needed //@param ret The output buffer to put \LaTeXe{} into.\\
\hyperlink{org.stfm.texdoclet.TableInfo.startHeadCol(javax.swing.text.MutableAttributeSet)}{{\bf startHeadCol(MutableAttributeSet)}} Starts a new Heading column, possibly closing the current column if needed.\\
\hyperlink{org.stfm.texdoclet.TableInfo.startRow(javax.swing.text.MutableAttributeSet)}{{\bf startRow(MutableAttributeSet)}} Starts a new row, possibly closing the current row if needed //@param ret The output buffer to put \LaTeX{} into.\\
\hyperlink{org.stfm.texdoclet.TableInfo.startTable(java.lang.StringBuffer, javax.swing.text.MutableAttributeSet)}{{\bf startTable(StringBuffer, MutableAttributeSet)}} Constructs a new table object and starts processing of the table by scanning the \texttt{\small \textless table\textgreater } passed to count columns.\\
\end{verse}
}
\subsection{Constructors}{
\vskip -2em
\begin{itemize}
\item{ 
\index{TableInfo()}
\hypertarget{org.stfm.texdoclet.TableInfo()}{{\bf  TableInfo}\\}
\texttt{public\ {\bf  TableInfo}()
\label{org.stfm.texdoclet.TableInfo()}}%end signature
}%end item
\end{itemize}
}
\subsection{Methods}{
\vskip -2em
\begin{itemize}
\item{ 
\index{endCol()}
\hypertarget{org.stfm.texdoclet.TableInfo.endCol()}{{\bf  endCol}\\}
\texttt{public void\ {\bf  endCol}()
\label{org.stfm.texdoclet.TableInfo.endCol()}}%end signature
\begin{itemize}
\item{
{\bf  Description}

Ends the current column. //@param ret The output buffer to put \LaTeXe{} into.
}
\end{itemize}
}%end item
\item{ 
\index{endRow()}
\hypertarget{org.stfm.texdoclet.TableInfo.endRow()}{{\bf  endRow}\\}
\texttt{public void\ {\bf  endRow}()
\label{org.stfm.texdoclet.TableInfo.endRow()}}%end signature
\begin{itemize}
\item{
{\bf  Description}

Ends the current row. // @param ret The output buffer to put \LaTeXe{} into.
}
\end{itemize}
}%end item
\item{ 
\index{endTable()}
\hypertarget{org.stfm.texdoclet.TableInfo.endTable()}{{\bf  endTable}\\}
\texttt{public java.lang.StringBuffer\ {\bf  endTable}()
\label{org.stfm.texdoclet.TableInfo.endTable()}}%end signature
\begin{itemize}
\item{
{\bf  Description}

Ends the table, closing the last row as needed
}
\end{itemize}
}%end item
\item{ 
\index{startCol(MutableAttributeSet)}
\hypertarget{org.stfm.texdoclet.TableInfo.startCol(javax.swing.text.MutableAttributeSet)}{{\bf  startCol}\\}
\texttt{public void\ {\bf  startCol}(\texttt{javax.swing.text.MutableAttributeSet} {\bf  attrSet})
\label{org.stfm.texdoclet.TableInfo.startCol(javax.swing.text.MutableAttributeSet)}}%end signature
\begin{itemize}
\item{
{\bf  Description}

Starts a new column, possibly closing the current column if needed //@param ret The output buffer to put \LaTeXe{} into. //@param p the properties from the \texttt{\small \textless td\textgreater } tag
}
\end{itemize}
}%end item
\item{ 
\index{startHeadCol(MutableAttributeSet)}
\hypertarget{org.stfm.texdoclet.TableInfo.startHeadCol(javax.swing.text.MutableAttributeSet)}{{\bf  startHeadCol}\\}
\texttt{public void\ {\bf  startHeadCol}(\texttt{javax.swing.text.MutableAttributeSet} {\bf  attrSet})
\label{org.stfm.texdoclet.TableInfo.startHeadCol(javax.swing.text.MutableAttributeSet)}}%end signature
\begin{itemize}
\item{
{\bf  Description}

Starts a new Heading column, possibly closing the current column if needed. A Heading column has a Bold Face font directive around it. //@param ret The output buffer to put \LaTeXe{} into. //@param p The properties from the \texttt{\small \textless th\textgreater } tag
}
\end{itemize}
}%end item
\item{ 
\index{startRow(MutableAttributeSet)}
\hypertarget{org.stfm.texdoclet.TableInfo.startRow(javax.swing.text.MutableAttributeSet)}{{\bf  startRow}\\}
\texttt{public void\ {\bf  startRow}(\texttt{javax.swing.text.MutableAttributeSet} {\bf  attrSet})
\label{org.stfm.texdoclet.TableInfo.startRow(javax.swing.text.MutableAttributeSet)}}%end signature
\begin{itemize}
\item{
{\bf  Description}

Starts a new row, possibly closing the current row if needed //@param ret The output buffer to put \LaTeX{} into. //@param p The properties from the \texttt{\small \textless tr\textgreater } tag
}
\end{itemize}
}%end item
\item{ 
\index{startTable(StringBuffer, MutableAttributeSet)}
\hypertarget{org.stfm.texdoclet.TableInfo.startTable(java.lang.StringBuffer, javax.swing.text.MutableAttributeSet)}{{\bf  startTable}\\}
\texttt{public java.lang.StringBuffer\ {\bf  startTable}(\texttt{java.lang.StringBuffer} {\bf  org},
\texttt{javax.swing.text.MutableAttributeSet} {\bf  attrSet})
\label{org.stfm.texdoclet.TableInfo.startTable(java.lang.StringBuffer, javax.swing.text.MutableAttributeSet)}}%end signature
\begin{itemize}
\item{
{\bf  Description}

Constructs a new table object and starts processing of the table by scanning the \texttt{\small \textless table\textgreater } passed to count columns. //@param p // properties found on the \texttt{\small \textless table\textgreater } tag //@param ret // the result buffer that will contain the output //@param table // the input string that has the entire table definition in it. //@param off // the offset into \texttt{\small \textless table\textgreater } where scanning // should start
}
\end{itemize}
}%end item
\end{itemize}
}
}
\section{\label{org.stfm.texdoclet.TestFilter}\index{TestFilter}Class TestFilter}{
\hypertarget{org.stfm.texdoclet.TestFilter}{}\vskip .1in 
This class filters out classes beginning with "Test" when applied to the Doclet.\vskip .1in 
\subsection{Declaration}{
\small public class TestFilter
\\ {\bf  extends} java.lang.Object
\refdefined{java.lang.Object}\\ {\bf  implements} 
ClassFilter}
\subsection{Constructor summary}{
\begin{verse}
\hyperlink{org.stfm.texdoclet.TestFilter()}{{\bf TestFilter()}} \\
\end{verse}
}
\subsection{Method summary}{
\begin{verse}
\hyperlink{org.stfm.texdoclet.TestFilter.includeClass(com.sun.javadoc.ClassDoc)}{{\bf includeClass(ClassDoc)}} Returns false if class name starts with "Test".\\
\end{verse}
}
\subsection{Constructors}{
\vskip -2em
\begin{itemize}
\item{ 
\index{TestFilter()}
\hypertarget{org.stfm.texdoclet.TestFilter()}{{\bf  TestFilter}\\}
\texttt{public\ {\bf  TestFilter}()
\label{org.stfm.texdoclet.TestFilter()}}%end signature
}%end item
\end{itemize}
}
\subsection{Methods}{
\vskip -2em
\begin{itemize}
\item{ 
\index{includeClass(ClassDoc)}
\hypertarget{org.stfm.texdoclet.TestFilter.includeClass(com.sun.javadoc.ClassDoc)}{{\bf  includeClass}\\}
\texttt{public boolean\ {\bf  includeClass}(\texttt{com.sun.javadoc.ClassDoc} {\bf  cd})
\label{org.stfm.texdoclet.TestFilter.includeClass(com.sun.javadoc.ClassDoc)}}%end signature
\begin{itemize}
\item{
{\bf  Description}

Returns false if class name starts with "Test".
}
\end{itemize}
}%end item
\end{itemize}
}
}
\section{\label{org.stfm.texdoclet.TeXDoclet}\index{TeXDoclet}Class TeXDoclet}{
\hypertarget{org.stfm.texdoclet.TeXDoclet}{}\vskip .1in 
This class provides a Java \texttt{\small javadoc} Doclet which generates a \LaTeXe{} document out of the java classes that it is used on. This is convenient for creating printable documentation complete with cross reference information.\subsection*{Supported HTML tags}
\begin{itemize}\item[\textless a\textgreater ]{including an additional attribut "doprinturl". Since the output of the doclet should be printable, the href attribut of tags is printed in parentheses following the link if attribut "doprinturl" is set. Sometimes this is undesirable, and omitting "doprinturl" attribut will prevent this. }\item[\textless dl\textgreater ]{with the associated \textless dt\textgreater \textless dd\textgreater \textless /dl\textgreater  tags}\item[\textless p\textgreater ]{but not align=center...yet}\item[\textless br\textgreater ]{but not clear=xxx}\item[\textless table\textgreater ]{including all the associated \textless td\textgreater \textless th\textgreater \textless tr\textgreater \textless /td\textgreater \textless /th\textgreater \textless /tr\textgreater }\item[\textless ol\textgreater ]{ordered lists}\item[\textless ul\textgreater ]{unordered lists}\item[\textless font\textgreater ]{font coloring}\item[\textless pre\textgreater ]{preformatted text}\item[\textless code\textgreater ]{fixed point fonts}\item[\textless i\textgreater ]{italized fonts}\item[\textless b\textgreater ]{bold fonts}\item[\textless sub\textgreater ]{subscript}\item[\textless sup\textgreater ]{superscript}\item[\textless center\textgreater ]{center}\item[\textless img\textgreater ]{image located in java sources (\textless img src="package path/image name"\textgreater )
\begin{itemize}\item[1. example]{converted from JPG: \mbox{\includegraphics[width=118pt, height=44pt]{texdoclet_images/pngimage0.png}}}\item[2. example]{converted from GIF: \mbox{\includegraphics[width=118pt, height=44pt]{texdoclet_images/pngimage1.png}}}
\end{itemize}
}\item[\textless img\textgreater ]{image located in the www: (see image at \hyperref{http://upload.wikimedia.org/wikipedia/commons/9/92/LaTeX_logo.svg}{}{}{http://upload.wikimedia.org/wikipedia/commons/9/92/LaTeX\_logo.svg})}
\end{itemize}
\subsection*{Extra tags}\subsubsection*{\textless TEX\textgreater }A new tag is defined: \texttt{\small \textless TEX\textgreater }. This tag is useful for passing \TeX{} code directly to the \TeX{} compiler. The following code:\texttt{\small
\mbox{}\newline \phantom{ }}\mbox{}\newline
\texttt{\small \phantom{ }\phantom{ }\textless TEX\phantom{ }txt="\textbackslash \lbrack\ F\textbackslash left(\phantom{ }x\phantom{ }\textbackslash right)\phantom{ }=\phantom{ }\textbackslash int\_\{\ -\phantom{ }\textbackslash infty\phantom{ }\}$\wedge$x\phantom{ }\{\textbackslash frac\{1\}\{\{\textbackslash sqrt\phantom{ }\{2\textbackslash pi\phantom{ }\}}\mbox{}\newline
\texttt{\small \phantom{ }\phantom{ }\phantom{ }\phantom{ }\phantom{ }\phantom{ }\phantom{ }\phantom{ }\phantom{ }\phantom{ }\phantom{ }\phantom{ }\phantom{ }\phantom{ }\phantom{ }\}\}e$\wedge$\{\ -\phantom{ }\textbackslash frac\{\{z$\wedge$2\phantom{ }\}\}\{2\}\}\ dz\}\ \textbackslash \rbrack "\textgreater }\mbox{}\newline
\texttt{\small \phantom{ }\phantom{ }\textless BR\textgreater \textless BR\textgreater \textless B\textgreater This\phantom{ }alternative\phantom{ }text\phantom{ }will\phantom{ }appear\phantom{ }if\phantom{ }the\phantom{ }javadoc/HTML\phantom{ }is\phantom{ }parsed}\mbox{}\newline
\texttt{\small \phantom{ }\phantom{ }by\phantom{ }any\phantom{ }other\phantom{ }doclet/browser\textless /B\textgreater \textless BR\textgreater \textless BR\textgreater \textless /TEX\textgreater }\mbox{}\newline
\texttt{\small \phantom{ }}


will produce the following result: \[ F\left( x \right) = \int_{ -
 \infty }^x {\frac{1}{{\sqrt {2\pi } }}e^{ - \frac{{z^2 }}{2}} dz} \] The "alternative" text is ignored by the TeXDoclet, but useful if you want to use both the TeXDoclet and a regular HTML based doclet.\subsubsection*{\textless PRE format="markdown"\textgreater }Instead of writing your java documentation in often hard to read HTML code you can make use of \hyperref{http://en.wikipedia.org/wiki/Markdown}{}{}{Markdown} syntax. The HTML \texttt{\small \textless PRE\textgreater } tag is used therefore to prevent your IDE from automatically reordering your Markdown documentation text. Markdown parsing is based on the \hyperref{https://github.com/sirthias/pegdown}{}{}{Pegdown} implementation. The following code :\texttt{\small
\mbox{}\newline \phantom{ }}\mbox{}\newline
\texttt{\small \phantom{ }\textless PRE\phantom{ }format="markdown"\textgreater }\mbox{}\newline
\texttt{\small \phantom{ }}\mbox{}\newline
\texttt{\small \phantom{ }some\phantom{ }text\phantom{ }some\phantom{ }text\phantom{ }some\phantom{ }text\phantom{ }some\phantom{ }text\phantom{ }some\phantom{ }text\phantom{ }some\phantom{ }text\phantom{ }some\phantom{ }text\phantom{ }}\mbox{}\newline
\texttt{\small \phantom{ }}\mbox{}\newline
\texttt{\small \phantom{ }\#\#\#\#\#\ Lists}\mbox{}\newline
\texttt{\small \phantom{ }}\mbox{}\newline
\texttt{\small \phantom{ }-\phantom{ }item1}\mbox{}\newline
\texttt{\small \phantom{ }\phantom{ }\phantom{ }\phantom{ }\phantom{ }1.\phantom{ }item11}\mbox{}\newline
\texttt{\small \phantom{ }\phantom{ }\phantom{ }\phantom{ }\phantom{ }2.\phantom{ }item12}\mbox{}\newline
\texttt{\small \phantom{ }-\phantom{ }item1}\mbox{}\newline
\texttt{\small \phantom{ }}\mbox{}\newline
\texttt{\small \phantom{ }\#\#\#\#\#\ Text\phantom{ }formatting}\mbox{}\newline
\texttt{\small \phantom{ }}\mbox{}\newline
\texttt{\small \phantom{ }\_emphasis\_\ and\phantom{ }\_\_strong\_\_\ and\phantom{ }some\phantom{ }`code`\phantom{ }:}\mbox{}\newline
\texttt{\small \phantom{ }}\mbox{}\newline
\texttt{\small \phantom{ }\phantom{ }\phantom{ }\phantom{ }\phantom{ }code\phantom{ }line\phantom{ }1}\mbox{}\newline
\texttt{\small \phantom{ }\phantom{ }\phantom{ }\phantom{ }\phantom{ }code\phantom{ }line\phantom{ }2}\mbox{}\newline
\texttt{\small \phantom{ }\phantom{ }\phantom{ }\phantom{ }\phantom{ }}\mbox{}\newline
\texttt{\small \phantom{ }some\phantom{ }text\phantom{ }some\phantom{ }text\phantom{ }some\phantom{ }text\phantom{ }some\phantom{ }text\phantom{ }some\phantom{ }text\phantom{ }some\phantom{ }text\phantom{ }some\phantom{ }text}\mbox{}\newline
\texttt{\small \phantom{ }}\mbox{}\newline
\texttt{\small \phantom{ }\textless PRE\textgreater }\mbox{}\newline
\texttt{\small \phantom{ }}\mbox{}\newline
\texttt{\small \phantom{ }}
will produce the following :\mbox{}\newline 



some text some text some text some text some text some text some text\subsubsection*{Lists}\begin{itemize}
\item{\vskip -.8ex item1
\begin{enumerate}
\item{\vskip -.8ex item11}
\item{\vskip -.8ex item12}
\end{enumerate}
}
\item{\vskip -.8ex item1}
\end{itemize}
\subsubsection*{Text formatting}

\textit{ emphasis} and {\bf strong} and some \texttt{\small code} :\texttt{\small
\mbox{}\newline \texttt{\small code\phantom{ }line\phantom{ }1}\mbox{}\newline
\texttt{\small code\phantom{ }line\phantom{ }2}}


some text some text some text some text some text some text some text\vskip .1in 
\subsection{See also}{}

  \begin{list}{-- }{\setlength{\itemsep}{0cm}\setlength{\parsep}{0cm}}
\item{ \texttt{\hyperlink{org.stfm.texdoclet.HTMLtoLaTeXBackEnd}{HTMLtoLaTeXBackEnd}} {\small 
\refdefined{org.stfm.texdoclet.HTMLtoLaTeXBackEnd}}%end
} 
\item{ \texttt{\hyperlink{org.stfm.texdoclet.TeXDoclet.start(com.sun.javadoc.RootDoc)}{TeXDoclet.start(RootDoc)}} {\small 
\refdefined{org.stfm.texdoclet.TeXDoclet.start(com.sun.javadoc.RootDoc)}}%end
} 
  \end{list}
\subsection{Declaration}{
\small public class TeXDoclet
\\ {\bf  extends} com.sun.javadoc.Doclet
\refdefined{com.sun.javadoc.Doclet}}
\subsection{Field summary}{
\begin{verse}
\hyperlink{org.stfm.texdoclet.TeXDoclet.BOLD}{{\bf BOLD}} \\
\hyperlink{org.stfm.texdoclet.TeXDoclet.CHAPTER_LEVEL}{{\bf CHAPTER\_LEVEL}} \\
\hyperlink{org.stfm.texdoclet.TeXDoclet.ITALIC}{{\bf ITALIC}} \\
\hyperlink{org.stfm.texdoclet.TeXDoclet.os}{{\bf os}} Writer for writing to output file\\
\hyperlink{org.stfm.texdoclet.TeXDoclet.SECTION_LEVEL}{{\bf SECTION\_LEVEL}} \\
\hyperlink{org.stfm.texdoclet.TeXDoclet.SUBSECTION_LEVEL}{{\bf SUBSECTION\_LEVEL}} \\
\hyperlink{org.stfm.texdoclet.TeXDoclet.TRUETYPE}{{\bf TRUETYPE}} \\
\end{verse}
}
\subsection{Constructor summary}{
\begin{verse}
\hyperlink{org.stfm.texdoclet.TeXDoclet()}{{\bf TeXDoclet()}} \\
\end{verse}
}
\subsection{Method summary}{
\begin{verse}
\hyperlink{org.stfm.texdoclet.TeXDoclet.main(java.lang.String[])}{{\bf main(String\lbrack \rbrack )}} \\
\hyperlink{org.stfm.texdoclet.TeXDoclet.optionLength(java.lang.String)}{{\bf optionLength(String)}} Doclet class method that returns how many arguments would be consumed if \texttt{\small option} is a recognized option.\\
\hyperlink{org.stfm.texdoclet.TeXDoclet.start(com.sun.javadoc.RootDoc)}{{\bf start(RootDoc)}} Doclet class method that is called by the framework to format the entire document\\
\hyperlink{org.stfm.texdoclet.TeXDoclet.validOptions(java.lang.String[][], com.sun.javadoc.DocErrorReporter)}{{\bf validOptions(String\lbrack \rbrack \lbrack \rbrack , DocErrorReporter)}} Doclet class method that checks the passed options and their arguments for validity.\\
\end{verse}
}
\subsection{Fields}{
\begin{itemize}
\item{
\index{SECTION\_LEVEL}
\label{org.stfm.texdoclet.TeXDoclet.SECTION_LEVEL}\hypertarget{org.stfm.texdoclet.TeXDoclet.SECTION_LEVEL}{public static final java.lang.String {\bf  SECTION\_LEVEL}}
}
\item{
\index{CHAPTER\_LEVEL}
\label{org.stfm.texdoclet.TeXDoclet.CHAPTER_LEVEL}\hypertarget{org.stfm.texdoclet.TeXDoclet.CHAPTER_LEVEL}{public static final java.lang.String {\bf  CHAPTER\_LEVEL}}
}
\item{
\index{SUBSECTION\_LEVEL}
\label{org.stfm.texdoclet.TeXDoclet.SUBSECTION_LEVEL}\hypertarget{org.stfm.texdoclet.TeXDoclet.SUBSECTION_LEVEL}{public static final java.lang.String {\bf  SUBSECTION\_LEVEL}}
}
\item{
\index{BOLD}
\label{org.stfm.texdoclet.TeXDoclet.BOLD}\hypertarget{org.stfm.texdoclet.TeXDoclet.BOLD}{public static final java.lang.String {\bf  BOLD}}
}
\item{
\index{TRUETYPE}
\label{org.stfm.texdoclet.TeXDoclet.TRUETYPE}\hypertarget{org.stfm.texdoclet.TeXDoclet.TRUETYPE}{public static final java.lang.String {\bf  TRUETYPE}}
}
\item{
\index{ITALIC}
\label{org.stfm.texdoclet.TeXDoclet.ITALIC}\hypertarget{org.stfm.texdoclet.TeXDoclet.ITALIC}{public static final java.lang.String {\bf  ITALIC}}
}
\item{
\index{os}
\label{org.stfm.texdoclet.TeXDoclet.os}\hypertarget{org.stfm.texdoclet.TeXDoclet.os}{public static java.io.PrintWriter {\bf  os}}
\begin{itemize}
\item{\vskip -.9ex 
Writer for writing to output file}
\end{itemize}
}
\end{itemize}
}
\subsection{Constructors}{
\vskip -2em
\begin{itemize}
\item{ 
\index{TeXDoclet()}
\hypertarget{org.stfm.texdoclet.TeXDoclet()}{{\bf  TeXDoclet}\\}
\texttt{public\ {\bf  TeXDoclet}()
\label{org.stfm.texdoclet.TeXDoclet()}}%end signature
}%end item
\end{itemize}
}
\subsection{Methods}{
\vskip -2em
\begin{itemize}
\item{ 
\index{main(String\lbrack \rbrack )}
\hypertarget{org.stfm.texdoclet.TeXDoclet.main(java.lang.String[])}{{\bf  main}\\}
\texttt{public static void\ {\bf  main}(\texttt{java.lang.String\lbrack \rbrack } {\bf  args})
\label{org.stfm.texdoclet.TeXDoclet.main(java.lang.String[])}}%end signature
}%end item
\item{ 
\index{optionLength(String)}
\hypertarget{org.stfm.texdoclet.TeXDoclet.optionLength(java.lang.String)}{{\bf  optionLength}\\}
\texttt{public static int\ {\bf  optionLength}(\texttt{java.lang.String} {\bf  option})
\label{org.stfm.texdoclet.TeXDoclet.optionLength(java.lang.String)}}%end signature
\begin{itemize}
\item{
{\bf  Description}

Doclet class method that returns how many arguments would be consumed if \texttt{\small option} is a recognized option.
}
\item{
{\bf  Parameters}
  \begin{itemize}
   \item{
\texttt{option} -- the option to check}
  \end{itemize}
}%end item
\end{itemize}
}%end item
\item{ 
\index{start(RootDoc)}
\hypertarget{org.stfm.texdoclet.TeXDoclet.start(com.sun.javadoc.RootDoc)}{{\bf  start}\\}
\texttt{public static boolean\ {\bf  start}(\texttt{com.sun.javadoc.RootDoc} {\bf  root})
\label{org.stfm.texdoclet.TeXDoclet.start(com.sun.javadoc.RootDoc)}}%end signature
\begin{itemize}
\item{
{\bf  Description}

Doclet class method that is called by the framework to format the entire document
}
\item{
{\bf  Parameters}
  \begin{itemize}
   \item{
\texttt{root} -- the root of the starting document}
  \end{itemize}
}%end item
\end{itemize}
}%end item
\item{ 
\index{validOptions(String\lbrack \rbrack \lbrack \rbrack , DocErrorReporter)}
\hypertarget{org.stfm.texdoclet.TeXDoclet.validOptions(java.lang.String[][], com.sun.javadoc.DocErrorReporter)}{{\bf  validOptions}\\}
\texttt{public static boolean\ {\bf  validOptions}(\texttt{java.lang.String\lbrack \rbrack \lbrack \rbrack } {\bf  args},
\texttt{com.sun.javadoc.DocErrorReporter} {\bf  err})
\label{org.stfm.texdoclet.TeXDoclet.validOptions(java.lang.String[][], com.sun.javadoc.DocErrorReporter)}}%end signature
\begin{itemize}
\item{
{\bf  Description}

Doclet class method that checks the passed options and their arguments for validity.
}
\item{
{\bf  Parameters}
  \begin{itemize}
   \item{
\texttt{args} -- the arguments to check}
   \item{
\texttt{err} -- the interface to use for reporting errors}
  \end{itemize}
}%end item
\end{itemize}
}%end item
\end{itemize}
}
\subsection{Members inherited from class Doclet }{
\texttt{com.sun.javadoc.Doclet} {\small 
\refdefined{com.sun.javadoc.Doclet}}
{\small 

languageVersion, optionLength, start, validOptions}
}
}
\begin{appendix}
\chapter{File appendix\_a.html}{
 \label{appendix1}


{\bf Appendix A content}

content of file doc-files/appendix\_a.html
}
\chapter{File appendix\_b.txt}{
 \label{appendix2}
content of file doc-files/appendix\_b.txt
}
\end{appendix}
\printindex
\end{document}
